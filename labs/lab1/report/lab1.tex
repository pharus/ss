\documentclass[12pt,a4paper]{article}
\usepackage[utf8]{inputenc}
\usepackage[portuguese]{babel}
\usepackage[T1]{fontenc}
\usepackage{amsmath}
\usepackage{amsfonts}
\usepackage{amssymb}
\usepackage{graphicx}
\usepackage{fourier}
\usepackage[bookmarks=true, hidelinks]{hyperref}
\author{Carlos Pinto Machado}
\title{
	Sinais e Sistemas - 1º Trabalho Prático \\
	Conceitos básicos sobre sinais e sistemas
}

\setcounter{section}{1}

\begin{document}

\maketitle

\tableofcontents

\newpage

\section{Sinais}
\subsection{Notas Musicais}
No guia, explicitou-se algumas propriedades das notas musicais, que serão 
relevantes na próxima subsecção.
Essas são:

\begin{itemize}
\item A escala musical é logarítmica, que implica uma razão de 2:1 entre cada 
oitava.
\item As diferenças entre as notas, por meio tom, são de $\sqrt[12]{2}$.
\item A frequência de $\text{Lá}_4$ é de 440 Hz.
\end{itemize}

\subsubsection{Cálculo das Frequências}
\begin{itemize}
\item $Sol_4$ é 1 tom abaixo de $\text{Lá}_4$.	
	\begin{equation}
		Sol_4 = \text{Lá}_4 \; (\sqrt[12]{2})^{-2} \approx 391.995 Hz
	\end{equation}
	
\item $Sol_4^\#$ é $\frac{1}{2}$ tom abaixo de $\text{Lá}_4$.	
	\begin{equation}
		Sol_4^\# = \text{Lá}_4 \; (\sqrt[12]{2})^{-1} \approx 415.305 Hz
	\end{equation}
	
\item $Si_4$ é 1 tom acima de $\text{Lá}_4$.	
	\begin{equation}
		Si_4 = \text{Lá}_4 \; (\sqrt[12]{2})^2 \approx 493.883 Hz
	\end{equation}
	
\item $\text{Dó}_5$ é $\frac{3}{2}$ tom acima de $\text{Lá}_4$.	
	\begin{equation}
		\text{Dó}_5 = _4 \; (\sqrt[12]{2})^{-2} \approx 523.251 Hz
	\end{equation}
\end{itemize}

\newpage

\section{Sistemas}

Tendo o sistema, cuja relação entrada-saída é dada por

\begin{equation}
	\begin{split}
		y(t) &= x(t) + A \; x(t-t_0), \\
		x(t) &- entrada \\		
		y(t) &- \text{saída}	\\
		A &- \text{amplitude de sinal atrasado, neste caso 0.7} \\
		t_0 &- \text{atraso do segundo sinal, neste caso 0.3}
	\end{split}
\end{equation}

\subsection{Linearidade e Invariância no tempo}
Classificação do sistema quanto à linearidade e invariância no tempo.

\subsubsection{Linearidade}
Para averiguar a linearidade de um sistema, vamos supor uma combinação linear
de dois sinais, como a entrada do sistema.

\begin{equation}
	x(t) = \alpha \; x_1(t) + \beta \; x_2(t) \;, \alpha, \; \beta \in \mathbb{R}
\end{equation}

Caso a resposta, a esta entrada seja a soma das repostas a cada um dos sinais constituintes, podemos deduzir que o sistema é linear.

\begin{equation}
	\begin{split}
		y(t) &= x(t) + A \; x(t - t_0) \\
		&= (\alpha \; x_1(t) + \beta \; x_2(t)) + A
		(\alpha \; x_1(t-t_0) + \beta \; x_2(t-t_0)) \\
		&= (\alpha \; x_1(t) + A \alpha \; x_1(t-t_0)) + 
		(\beta \; x_2(t) + A \beta \; x_2(t-t_0)) \\
		&= \alpha \; (x_1(t) + A \; x_1(t-t_0)) + 
		\beta \; (x_2(t) + A \; x_2(t-t_0)) \\
		&= \alpha \; y_1(t) + \beta \; y_2(t) \\ \\
		y_1(t) &- \text{saída do sistema, com }x(t) = x_1(t) \\
		y_2(t) &- \text{saída do sistema, com }x(t) = x_2(t)
	\end{split}
\end{equation}

Com esta dedução, podemos concluir que o sistema é linear.

\newpage
\subsubsection{Invariância no tempo}

Com o objectivo de deduzir se o sistema é invariante no tempo, teremos de
considerar um sinal com uma dada fase, e testar se o sinal comporta-se da 
mesma forma.

\begin{equation}
	x_3(t) = x(t - t_0'), \; t_0' > 0
\end{equation}

\begin{equation}
	\begin{split}	
		y_3(t) &= x_3(t) + A \; x_3(t - t_0') \\
		&= x(t - t_0') + A \; x(t - t_0' - t_0) \\
		&= y(t - t_0')
	\end{split}
\end{equation}

Desta forma, concluímos que o sistema é invariante no tempo.

\newpage
\subsection{Resposta ao Impulso Unitário}

Determinação da resposta do sistema ao impulso unitário.

Considerando o impulso unitário,
\begin{equation}
	\delta (t) =
	\begin{cases}
		\; 1 \;, t = 0 \\
		\; 0 \;, t \neq 0
	\end{cases}, t \in \mathbb{R},
\end{equation}

e o sistema, em questão, podemos concluir que a resposta do sistema apenas
assume 3 valores distintos, tendo em conta o atraso $t_0$ na saída do sistema.
Estes podem-se reduzir da seguinte forma, em 5 etapas.

\begin{enumerate}
	\item Antes do impulso em $t = 0$, que é 0
	\item Em $t=0$, durante o impulso, que é 1
	\item Em $t \in ]0, \; t_0[$, que é 0	
	\item Em $t = t_0$, durante a componente que tem fase, que é A
	\item Em $t > t_0$, que é 0
\end{enumerate}

\begin{equation}
	\begin{split}
		y(t) &= x(t) + A \; x(t - t_0) \\
		y(t) &= 
		\begin{cases}
			\; 1 \; &, t = 0 \\
			\; A \;&, t = t_0 \\
			\; 0 \; &, t \not \in \; \{0, \; t_0\}
		\end{cases}
	\end{split}
\end{equation}



\newpage
\subsection{Memória, Causalidade e Estabilidade}
Classificação do sistema a determinadas propriedades, nomeadamente memória, 
causalidade e estabilidade.

\subsubsection{Memória}
O sistema possui memória, caso $t_0 \neq 0$. Neste contexto possui memória,
porque $t_0 = 0.3$, que implica que depende de uma entrada de outro instante,
sem ser o presente aquando os instante da resposta.

\subsubsection{Causalidade}
O sistema é causal, caso $t_0 \geq 0$, que implica a existência de um atraso,
ou a inexistência de fase, ou seja, não possui a previsão da entrada. Sendo
$t_0 = 0.3$, o sistema é causal, porque apenas depende das entradas presentes e 
passadas.

\subsubsection{Estabilidade}
O sistema é estável, caso para uma entrada limitada, tem uma saída limitada.
Dada à análise da resposta ao impulso unitário, podemos concluir que o sistema
é estável.

\newpage
\subsection{Resposta a um sinal com dois steps}
Determinação da resposta do sistema ao sinal $x_1(t) = u(t+1) - u(t-1)$.
Considerando o sinal de step,

\begin{equation}
	u(t) = 
	\begin{cases}
		\; 1 \;, t \geq 0 \\
		\; 0 \;, t < 0
	\end{cases}
	, t \in \mathbb{R},
\end{equation}

e o sinal step ser invariante no tempo, podemos deduzir que, dada a presença da
fase na resposta do sistema, possui 4 valores distintos, em 5 etapas diferentes.

\begin{enumerate}
	\item Antes do primeiro step, em $t < -1$, que é de 0
	\item Em $t \in [-1, \; 1 + t_0[$, que é de 1
	\item Em $t \in [-1 + t_0, \; 1[$, que é de 1 + A
	\item Em $t \in [1, \; 1 + t_0[$, que é de A
	\item Em $t \geq 1 + t_0$, que é de 0
\end{enumerate}

\begin{equation}
	y_1(t) = 
	\begin{cases}
		\; 0 \;   &, \; t \in ]-\infty,\; -1[ \; \vee \;[1 + t_0, \; + \infty [\\
		\; 1 \;   &, \; t \in [-1, -1 + t_0[ \\
		\; 1+A\;  &, \; t \in [-1 + t_0, \; 1[ \\
		\; A \;   &, \; t \in [1, \; 1 + t_0[
	\end{cases}
\end{equation}

\newpage
\subsection{Resposta a um sinal com um seno}
Por Responder

\end{document}


















