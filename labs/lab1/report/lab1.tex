\documentclass[12pt,a4paper]{article}
\usepackage[utf8]{inputenc}
\usepackage[portuguese]{babel}
\usepackage[T1]{fontenc}
\usepackage{amsmath}
\usepackage{amsfonts}
\usepackage{amssymb}
\usepackage{graphicx}
\usepackage{fourier}
\author{Carlos Pinto Machado}
\title{
	Sinais e Sistemas - 1º Trabalho Prático \\
	Conceitos básicos sobre sinais e sistemas
}

\setcounter{section}{1}

\begin{document}

\maketitle

\tableofcontents

\newpage

\section{Sinais}
\subsection{Notas Musicais}
No guia, explicitou-se algumas propriedades das notas musicais, que serão relevantes na próxima subsecção.
Essas são:

\begin{itemize}
\item A escala musical é logarítmica, que implica uma razão de 2:1 entre cada oitava.
\item As diferenças entre as notas, por meio tom, são de $\sqrt[12]{2}$.
\item A frequência de $Lá_4$ é de 440 Hz.
\end{itemize}

\subsubsection{Cálculo das Frequências}
\begin{itemize}
\item $Sol_4$ é 1 tom abaixo de $Lá_4$.	
	\begin{equation}
		Sol_4 = Lá_4 \; (\sqrt[12]{2})^{-2} \approx 391.995 Hz
	\end{equation}
	
\item $Sol_4^\#$ é $\frac{1}{2}$ tom abaixo de $Lá_4$.	
	\begin{equation}
		Sol_4^\# = Lá_4 \; (\sqrt[12]{2})^{-1} \approx 415.305 Hz
	\end{equation}
	
\item $Si_4$ é 1 tom acima de $Lá_4$.	
	\begin{equation}
		Si_4 = Lá_4 \; (\sqrt[12]{2})^2 \approx 493.883 Hz
	\end{equation}
	
\item $Dó_5$ é $\frac{3}{2}$ tom abaixo de $Lá_4$.	
	\begin{equation}
		Dó_5 = Lá_4 \; (\sqrt[12]{2})^{-2} \approx 523.251 Hz
	\end{equation}
\end{itemize}

\section{Sistemas}

\end{document}

